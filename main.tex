\documentclass[10pt, a4paper]{article}

% Fonts
\usepackage[english]{babel}
\usepackage{lmodern}
\usepackage[T1]{fontenc}

% Math, Physics
\usepackage{mathtools}
\usepackage{amsfonts}
\usepackage{physics, tensor}

% Margins and spacing
\usepackage{geometry}
\geometry{left=25mm, right=25mm, top=30mm, bottom=30mm}
\usepackage{setspace}
\singlespacing % or \onehalfspacing or \doublespacing

% Misc
\usepackage[none]{hyphenat}
\usepackage[hidelinks]{hyperref}
\usepackage{cleveref}

% Titles
\usepackage{titlesec}
\titleformat
{\section} % command
{\sffamily\bfseries\Large} % format
{\thesection.} % label
{1ex} % sep
{} % before-code
[] % after-code

% Title page
\usepackage{titling}
\pretitle{\LARGE\bfseries\sffamily\noindent}
\posttitle{
    \par
    \noindent\makebox[\linewidth]{\rule{\linewidth}{1pt}}
    \vspace{2ex}
}
\preauthor{
    \par\large\bfseries\sffamily\noindent
}
\postauthor{\par}
\predate{\par\vspace{1ex}}
\postdate{\par\vspace{2ex}}

% Headers and Footers
\usepackage{fancyhdr}
\pagestyle{fancy}
\fancyhead[L]{Title or Section}
\fancyhead[R]{Page \thepage}
\fancyfoot[C]{}

% References
\usepackage[backend=biber, style=numeric-comp, sorting=none]{biblatex}
\usepackage{csquotes}
\addbibresource{references/figures.bib}
\addbibresource{references/papers.bib}

% Figures
\usepackage{graphicx, float}
\graphicspath{{figures/}}
\usepackage{caption}
\usepackage{subcaption}

% Tables
\usepackage{booktabs}

% Definitions
\newcommand{\ii}{\mathrm{i}}

\title{Notes on finite-field quantization of BFSS}
\author{Rishi Raj}
\date{
    \noindent\textit{Sorbonne Université, Paris, FR}\\[0.2cm]
    \href{mailto:rishiraj.1012exp@gmail.com}{\texttt{rishi.raj@sorbonne-universite.fr}}
}

\numberwithin{equation}{section}

\renewenvironment{abstract}
 {\par\noindent\textbf{\abstractname}\ \ignorespaces\\[1ex]}
 {\par\medskip}

\begin{document}

\thispagestyle{empty}
\maketitle

\begin{abstract}
  We explore finite field quantization as a discrete framework for studying quantum mechanical and matrix models. 
  Starting with the finite field quantization of the harmonic oscillator, we extend the construction to the quartic oscillator as a nontrivial interacting example. 
  We then formulate finite field analogues of the gauged BFSS matrix model, making use of unitary groups over finite rings, and perform numerical simulations to probe their dynamics. 
  These results provide initial insights into the behaviour of matrix quantum mechanics under finite field regularizations.
\end{abstract}

\noindent\today

\tableofcontents

\setlength{\parindent}{1ex}
\setlength{\parskip}{2ex}
\raggedright

\section{Setup}
We have the BFSS matrix model consisting of $d$ Hermitian $N \times N$ matrices $X^i$ with the Hamiltonian
\begin{equation}
  H = \frac{1}{2} \tr \dot{X}^i \dot{X}^i - \frac{1}{4} \tr \comm{X^i}{X^j} \comm{X^i}{X^j}.
\end{equation}
The idea is to consider the subsector where every entry in $X^i$ is from $\mathbb{Z}_M[\ii]$ i.e.
\begin{equation}
  X^i_{A B} = a + \ii b, \quad a, b \in \mathbb{Z}_M.
\end{equation}
This effectively reduces the BFSS matrix model to a finite state system with $M^{2d N^2}$ states and the corresponding Hilbert space after quantization to a finite-dimensional one.

\subsection{Gauging}
\paragraph{Unitary groups over $\mathbb{Z}_M[\ii]$.} 
Let $M \in \mathbb{Z}_{>0}$ and consider the Gaussian extension ring
\[
  \mathbb{Z}_M[\ii] := \{\, a + b \ii \;\;|\;\; a,b \in \mathbb{Z}_M, \; \ii^2 = -1 \,\},
\]
equipped with the natural involution (complex conjugation) 
\[
  \overline{a+b\ii} = a - b\ii.
\]
This makes $\mathbb{Z}_M[\ii]$ into a finite ring with involution.

For $N \in \mathbb{Z}_{>0}$, the group of unitary matrices over $\mathbb{Z}_M[\ii]$ is defined as
\[
  U(N,M) \;:=\; \bigl\{\, U \in \mathrm{GL}_N(\mathbb{Z}_M[\ii]) \;\big|\; U^\dagger U = I \,\bigr\},
\]
where $U^\dagger = \overline{U}^{\,T}$ denotes the conjugate transpose.

\medskip
\noindent
Some remarks and subcases:
\begin{itemize}
  \item If $M$ is prime, then $\mathbb{Z}_M \cong \mathbb{F}_p$ is a finite field, and $\mathbb{Z}_M[\ii]$ is a quadratic extension of $\mathbb{F}_p$ (a field if $p \equiv 3 \pmod{4}$, or isomorphic to $\mathbb{F}_p \times \mathbb{F}_p$ if $p \equiv 1 \pmod{4}$). 
  In the field case, $U(N,M)$ coincides with the classical finite unitary group $U(N,p)$ preserving a Hermitian form over $\mathbb{F}_{p^2}$.

  \item If $M$ is composite, then $\mathbb{Z}_M[\ii]$ is not a field but a finite ring with zero divisors. 
  By the Chinese remainder theorem, one has a factorization
  \[
    \mathbb{Z}_M[\ii] \;\cong\; \prod_j \mathbb{Z}_{p_j^{k_j}}[\ii],
  \]
  so correspondingly
  \[
    U(N,M) \;\cong\; \prod_j U\!\bigl(N, p_j^{k_j}\bigr).
  \]

  \item Within $U(N,M)$, one defines the \emph{special unitary group} 
  \[
    SU(N,M) := \{\, U \in U(N,M) \;\;|\;\; \det(U) = 1 \,\}.
  \]
  Quotienting further by the scalar center yields the \emph{projective special unitary group} $PSU(N,M) := SU(N,M)/Z$.
\end{itemize}


\section{Finite field Quantum Mechanics}
Consider a particle with position $x$ in a finite lattice with lattice space $a$ and periodic boundary conditions
\begin{equation}
  x = \ell a, \quad \ell \in \mathbb{Z}, \qquad x \sim x + 2 \pi R, \quad 2\pi R = M a, \quad M \in \mathbb{Z},
\end{equation}
This effectively makes $\ell \in \mathbb{Z}_M$ and the momentum $p$ in the dual lattice
\begin{equation}
  p = \frac{2 \pi}{M a} n, \quad n \in \mathbb{Z}_M, \qquad p \sim p + \frac{2 \pi}{a}.
\end{equation}
We denote the position basis with $\ket{x_\ell}$ and is such that
\begin{equation}
  \bra{x_\ell}\ket{x_{\ell'}} = \delta_{\ell \ell'}, \quad \sum_{\ell \in \mathbb{Z}_M} \ket{x_\ell}\bra{x_\ell} = 1.
\end{equation}
We can also construct the momentum basis $\ket{p_n}$ as follows. We know
\begin{equation}
  \bra{x_\ell}\ket{p_n} = \mathcal{A} \, e^{\ii p x} = \mathcal{A} \, e^{\ii \frac{2 \pi}{M} \ell n}.
\end{equation}
The norm gives us
\begin{equation}
  \delta_{n n'} = \bra{p_{n'}} \ket{p_{n'}} = \sum_{\ell \in \mathbb{Z}_M} \bra{p_{n'}}\ket{x_\ell} \bra{x_\ell}\ket{p_n} = \abs{\mathcal{A}}^2 \sum_{\ell} e^{\frac{2\pi \ii}{M} \ell (n - n')} = \abs{\mathcal{A}}^2 \, M \, \delta_{n n'}.
\end{equation}
Thus, $\ket{p_n} = \frac{1}{\sqrt{M}} \sum_{\ell} e^{\frac{2\pi\ii}{M} \ell n} \ket{x_\ell}$. We can also work out the momentum operator in the position basis
\begin{equation}
  \hat{p} = \sum_{n \in \mathbb{Z}_M} p_n \ket{p_n} \bra{p_n} = \frac{2 \pi}{M^2 a} \, \sum_{n \ell \ell'} \, n  \, \exp(\frac{2 \pi \ii}{M} n (\ell - \ell')) \ket{x_\ell} \bra{x_{\ell'}}.
\end{equation}
It is easy to show
\begin{equation}
  \sum_{n \in \mathbb{Z}_M} n \exp(\frac{2 \pi \ii}{M} n (\ell - \ell')) = \begin{cases}
    \frac{M (M-1)}{2} & \ell = \ell', \\
    \frac{M}{e^{\frac{2\pi\ii}{M} (\ell - \ell')} - 1} & \ell \neq \ell'.
  \end{cases}
\end{equation}
Thus,
\begin{equation}
  \boxed{
  \hat{p} = \frac{2 \pi}{M a} \Biggl\{ \qty(\tfrac{M-1}{2}) \sum_{\ell} \ket{x_\ell} \bra{x_\ell} \ + \ \sum_{\ell \neq \ell'} \qty(\tfrac{1}{e^{\frac{2\pi\ii}{M} (\ell - \ell')} - 1}) \ket{x_\ell} \bra{x_{\ell'}} \Biggr\}.
  }
\end{equation}

We can check this by computing the translation operator
\begin{equation}
  \hat{T} = e^{\ii \hat{p} a}
\end{equation}


\printbibliography

\end{document}